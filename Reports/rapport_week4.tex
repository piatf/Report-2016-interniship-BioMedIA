\documentclass{report}
\usepackage[utf8]{inputenc}
\usepackage[top=0cm, bottom=0cm, left=2cm, right=2cm]{geometry}
\usepackage[francais]{babel}
\usepackage[T1]{fontenc}
\usepackage{graphicx}
\usepackage{subcaption}

\title{Rapport}
\author{François PIAT}
\date{WEEK 3 - }

\begin{document}

\maketitle

\chapter*{Profiling}

L'un des buts du stage est de ne plus utiliser la bibliothèque de programmation parallèle appelée TBB, en remplaçant ce qui sera necéssaire par ArrayFire, qui a une fonctionnalité semblable.
\\
Pour cela, on interviendra de 2 manières :\\
- Les fonctions où TBB est efficace : on remplace par l'équivalent d'AF quand cela est possible. \\
- Les fonctions où TBB n'est pas efficace : on supprime l'utilisation de TBB.
\paragraph{Valgrind}
Valgrind est une suite d'outils informatiques, incluant un debugger, un profiler, un analyseur de mémoire...
On utilise callgrind, le profiler pour analyser les performances de MIRTK à partir d'une étude des caches lors de l'éxecution des fonctions 

\paragraph{Stratégie} 
- Analyse sur un mode debug activé, puis comparaison entre tbb on et tbb off.  - On analyse les filtres en premier (*) \newline Fonctions analysées : \newline
\\close-image*
cut-brain
detect-edges
dilate-image*
downsample-image*
erode-image*
evaluate-similarity
evaluate-overlap
invert-dof
open-image*
reflect-image
register : (- rigid only
- affine only
- ffd only)
resample-image*
smooth-image*
transform-image*\\

Pour transform-image et resample-image, on fait un test sur chacun des modes d'interpolation suivants: Linéaire, NN, BSpline, Sinc et Gaussien.

\paragraph{Problèmes}
- Certains tests sont très long à réaliser, ce que je n'avait pas prévu. J'ai donc fais du multi-tasking, mais seulement à la fin des tests.
\newline - Les tests sur des programmes en multi-threading (TBB) n'arrivent pas à 100 pourcent dans Kcachegrind en cumulé. 

\chapter*{Bazar}
-Découverte de Jabref/Zotero, pour gérer les ressources documentaires.

\end{document}